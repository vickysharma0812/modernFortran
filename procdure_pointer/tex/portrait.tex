%%%%%%%%%%%%%%%%%%%%%%%%%%%%%%%%%%%%%%%%
% Class options                        %
%%%%%%%%%%%%%%%%%%%%%%%%%%%%%%%%%%%%%%%%
% Orientation:                         %
% portrait (default), landscape        %
%                                      %
% Paper size:                          %
% a0paper (default), a1paper, a2paper, %
% a3paper, a4paper, a5paper, a6paper   %
%                                      %
% Language:                            %
% english (default), norsk             %
%%%%%%%%%%%%%%%%%%%%%%%%%%%%%%%%%%%%%%%%
\documentclass{uioposter}


\usepackage{lipsum}                                % Dummy text
\usepackage[figwidth = 0.98\linewidth]{todonotes}  % Dummy image (and more!)
\usepackage[absolute, overlay]{textpos}            % Figure placement
\setlength{\TPHorizModule}{\paperwidth}
\setlength{\TPVertModule}{\paperheight}
\usepackage{listings}


\title{Understanding Procedure Pointers in Modern Fortran}
\author
{%
    Vikas Sharma \inst{1}
    %\and
    %Second Author\inst{2}
    %\and
    %Third Author\inst{1}
}
%% Optional:
\institute
{
    \inst{1} Civil Engineering Department \\
    IIT Bombay, Mumbai, India
    %\and
    %\inst{2} Department of Informatics
}
% Or:
%\institute{Contact information}


%% Remove footline:
\setbeamertemplate{footline}{}


\begin{document}
\begin{columns}[onlytextwidth]


\begin{column}{0.5\textwidth - 1.5cm}
    \begin{block}{Introduction}
	Procedure pointers are new addition to Modern Fortran. In a loose sense, it facilitates the use of function as variables. To understand them consider the following.

		(1) Ask use for some input, let's say \verb|n|
		\\
		(2) If \verb|n| equals to 1 then evaluate a function called \verb|func_1|
		and display its content
		\\
		(3) If \verb|n| equals to 2 then evaluate a function called \verb|func_2|
		and display its content
		\\
		\textit{I think, now you have got the idea about what we want to do. Here,
		ignore the simplicity of the present problem and focus on the intent.}
    \end{block}

	\begin{block}{How to define procedure pointers}
		\begin{verbatim}
			PROCEDURE( func ), POINTER :: f => NULL()
		\end{verbatim}

		Here \verb|func| denotes an abstract function (think it like a template) which informs the compiler about  the argument of the function and its results. An interface for \verb|func| is usually described by using \verb|INTERFACE| block or \verb|ABSTRACT INTERFACE| block. See the example given below:

		\begin{verbatim}
			ABSTRACT INTERFACE
			PURE FUNCTION func( x ) RESULT( Ans )
			IMPORT :: DFP
			REAL( DFP ), INTENT( IN ) :: x
			REAL( DFP ) :: Ans
			END FUNCTION func
			END INTERFACE
		\end{verbatim}
	\end{block}

\begin{block}{Application}
	In my research, I use procedure pointer in creating a automatic differentiation library. We can also use procedure pointers to pass functions and subroutine to another routines. These topics will be covered in my next post.
\end{block}

    %\begin{exampleblock}{Does it come in black?}
     %   Sure, use an \textbf{exampleblock}!
    %\end{exampleblock}

    %\begin{alertblock}{How do you make it pop?}
     %   Use an \alert{alertblock}!
    %\end{alertblock}

    %\begin{block}{Results}
     %   \lipsum[2]
       % \missingfigure{Striking imagery relevant to the research}
        %\unskip
    %\end{block}
\end{column}


\begin{column}{0.5\textwidth - 1.5cm}
    \begin{block}{Code}
    	\begin{verbatim}
    	PROGRAM main
    	USE, INTRINSIC :: ISO_FORTRAN_ENV, &
    	& ONLY : DFP=>real64, I4B=>int32
    	IMPLICIT NONE

    	ABSTRACT INTERFACE
    	PURE FUNCTION func( x ) RESULT( Ans )
    	IMPORT :: DFP
    	REAL( DFP ), INTENT( IN ) :: x
    	REAL( DFP ) :: Ans
    	END FUNCTION func
    	END INTERFACE

    	PROCEDURE( func ), POINTER :: f => NULL()
    	INTEGER( I4B ) :: n

    	WRITE( *, "(A)", ADVANCE="NO" ) &
    		& "Enter the value of case (n) :: "
    	READ( *, * ) n

    	SELECT CASE( n )
    	CASE( 1 )
    	f => func_1
    	CASE( 2 )
    	f => func_2
    	CASE( 3 )
    	f =>  func_3
    	CASE DEFAULT
    	f => func_1
    	END SELECT

    	WRITE( *, * ) f( 2.0_DFP )

    	CONTAINS
    	PURE FUNCTION func_1( x ) RESULT( Ans )
    	REAL( DFP ), INTENT( IN ) :: x
    	REAL( DFP ) :: Ans
    	Ans = SIN( x )
    	END FUNCTION func_1

    	PURE FUNCTION func_2( x ) RESULT( Ans )
    	REAL( DFP ), INTENT( IN ) :: x
    	REAL( DFP ) :: Ans
    	Ans = COS( x )
    	END FUNCTION func_2

    	PURE FUNCTION func_3( x ) RESULT( Ans )
    	REAL( DFP ), INTENT( IN ) :: x
    	REAL( DFP ) :: Ans
    	Ans = TAN( x )
    	END FUNCTION func_3
    	END PROGRAM main
    	\end{verbatim}
    \end{block}
\end{column}


\end{columns}

\end{document}